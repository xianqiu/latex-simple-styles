%!TEX program = xelatex

\documentclass{cls/simplecv}

\begin{document}

\name{林浩 LIN Hao}
\contact{\phone ~138 1234 5678  \quad \mail~ linhao@email.com }

\maketitle

\section{自我评价}

\textbf{本内容完全虚构,仅用于展示简历样式。}

资深全栈工程师,7年人工智能与分布式系统研发经验。擅长Python/Java技术栈,主导过3个百万级用户规模的AI产品落地,精通深度学习模型优化与高并发微服务架构设计。获ACM国际编程竞赛亚洲区金奖,发表IEEE/SCI论文5篇,持有6项技术专利。具备从0到1的技术攻关能力与跨部门协作经验,擅长将学术成果转化为工业级解决方案。

\section{项目经历}

\textbf{2022.03 -- 2023.06 \quad 基于深度学习的要干图像分类系统}
\begin{itemize}
	\item 技术栈:PyTorch/OpenCV/Django
	\item 构建ResNet-Transformer混合模型,实现98.2\%的耕地识别准确率
	\item 设计多线程数据预处理流水线,训练效率提升300\%
	\item 获2023年中国人工智能大会最佳应用创新奖
\end{itemize}

\textbf{2020.09 -- 2022.02 \quad 分布式电商推荐系统重构}
\begin{itemize}
	\item 技术栈:Spring Cloud/Redis/Flink
	\item 重构原有单体架构,QPS从1500提升至12000+
	\item 实现实时用户画像更新,推荐转化率提升27\%
\end{itemize}

\textbf{2019.01 -- 2020.06 \quad 医疗知识图谱可视化平台}
\begin{itemize}
	\item 技术栈:Neo4j/React/D3.js
	\item 构建包含530万实体关系的医疗图谱
	\item 开发动态关系发现算法,辅助诊断效率提升40\%
\end{itemize}

\section{教育经历}

\begin{tabular}{llll}
	2023.07 -- 2020.06 & 博士 & 联邦学习 & 清华大学 \\ 
	2014.09 -- 2017.06 & 硕士 & 分布式计算 & 浙江大学\\
	2010.09 -- 2014.06 & 本科 & 计算机科学与技术 & 南京大学
\end{tabular}

\section{工作单位}
\begin{tabular}{lll}
    2023.07 -- 今 & 首席 AI 工程师/ AutoML工具链 & 微软亚洲研究院 \\
    2020.07 -- 2023.06 & 高级研发工程师/推荐算法 & 阿里巴巴达摩院\\
    2017.07 -- 2020.06 & 全栈开发工程师 / 推荐算法 & 字节跳动\\
\end{tabular}

\section{个人技能}
\begin{tabular}{ll}
	{[语言]} & 精通 Python;熟悉 Java/C++;常用 Shell/Git \\
	{[框架]} & TensorFlow/Kubernetes/Spark \\
	{[工具]} & Gitlab CI/CD/Elasticsearch/Airflow \\
	{[领域]} & 机器学习系统设计/云原声架构/性能调优
\end{tabular}

\section{代表专利}

[1]《基于多模态融合的智能客服对话系统》(ZL202310123456.7)2023.05

[2]《区块链赋能的供应链溯源方法》(ZL202210654321.1)2022.11

[3]《无人机集群协同路径规划算法》(ZL202180765432.9)2021.09


\section{代表论文}
[1] Hao Lin. \emph{Federated Learning with Adaptive Differential Privacy}
IEEE TPAMI (IF=24.314), 2023.04.

[2] Hao Lin. \emph{Dynamic Graph Convolution for 3D Point Cloud Segmentation}. CVPR 2022 Oral.

[3] Hao Lin. \emph{A Survey of Edge Computing Architectures}. ACM Computing Surveys, 2020.12.

\end{document}