\documentclass{cls/simplereport}


\begin{document}
\title{A Simple \LaTeX~ Report Template}
\author{xianqiu}
\date{2024-05-08}
\maketitle
	
\section{列表样式}
	
什么是 TikZ/PGF?TikZ和PGF是一种用在 TeX 上的 CLI 绘图工具。

\begin{enumerate}
	\item TikZ 通过类编程的思想实现绘图,这种方式往往能够生成精确控制的函数图,常见的有PostScript、PGF、Asymptote、PSTricks等。
	\item Tikz 基于 PGF,PGF的全名是“portable graphics format”。
	\item TikZ的命名很有趣,采用的是递归式的取名:“TikZ ist kein Zeichenprogramm”(TikZ is not a drawing program)。类似的取名最出名的恐怕就是GNU(GNU is Not Unix)了。
\end{enumerate}

什么是 Beamer?Beamer 是一个用于制作演示文稿的 LaTeX 宏包,它可以将 LaTeX 文档转换为 PDF 格式的幻灯片。
Beamer 的主要特点和功能如下:

\begin{itemize}
	\item 用户只需了解基本的 \LaTeX 语法,就能快速上手 Beamer。
	\item 通过简单的命令,用户可以插入图像,创建项目符号列表,调整字体样式等,从而高效地创建高质量的幻灯片。
	\begin{itemize}
		\item 支持多种主题和样式,用户可以根据自己的喜好选择合适的主题。
		\item 支持多种语言,包括英语、德语、法语、西班牙语等。
		\item 支持多种格式,包括 PDF、HTML、LaTeX 等。
	\end{itemize}
	\item 无论是科研报告、教学课件,还是商务演示,Beamer 都是一个理想的选择。
	\end{itemize}

\subsection{示意图}

\begin{figure}[ht]
	\centering
	\begin{tikzpicture}
		\node[scale=0.75] at (0,  0) {\input{example-report-gfx/eg_look}};
	\end{tikzpicture}
\end{figure}

\section{表格}

\renewcommand{\figurename}{图} % figure prefix
\renewcommand{\tablename}{表} % table prefix
\renewcommand{\lstlistingname}{列表} % code prefix

\begin{table}[H]
	\centering
	\begin{tabular}{c|c|p{8cm}}
		\toprule
		\textbf{类别} & \textbf{玩法数量} & \textbf{明细} \\
		\midrule
		单商品 & 8 & 积分商品、秒杀、付费会员专享价、免费试用、限时购(限时限量购)、定金购、塞购物车、特价(限时不限量)\\
		\midrule
		多商品 & 7 & 多商品加价购、多商品满送商品、多商品满折、N元任选、套装价、多商品满件减、多商品满额减\\
		\midrule
		全场 & 3& 全场加价购、全场满送券、全场满额减\\
		\midrule
		特殊 & 2 & 拼团、众筹 \\
		\bottomrule
	\end{tabular}
	\caption{长文本自动换行}
	\label{tab:wf}
\end{table}

\begin{table}[H]
	\centering
	\begin{tabular}{|c|c|c|c|}
		\toprule
		\multirow{2}*{姓名}
		&\multicolumn{3}{c|}{成绩} \\ \cmidrule{2-4}
		& 语文 & 数学 & 英语\\ 
		\midrule
		大明 & 97 & 99 & 80 \\ 
		\midrule
		王二 & 85 & 92 & 95 \\
		\bottomrule
	\end{tabular}
	\caption{合并单元格}
\end{table}
	
\section{定理和公式}
	
\newtheorem{theorem}{定理}
\newtheorem{corollary}[theorem]{Corollary}
\newtheorem{lemma}[theorem]{Lemma}
\newtheorem{definition}[theorem]{Definition}

\begin{theorem}
	给定任意的非负整数 $n$,我们有
	$$(1+x)^n = \sum_{i=0}^n {n \choose i} x^i$$
\end{theorem}

\begin{theorem}
	给定任意的集合 $A$,$B$ 和 $C$,我们有
	$$ (A\cup B)-(C-A) = A \cup (B-C)$$
\end{theorem}

\begin{description}
	\item [求和]
	\begin{equation}
		e^x = 1 + x + \frac{x^2}{2} + \frac{x^3}{6} + \cdots = \sum_{n\geq 0} \frac{x^n}{n!}
	\end{equation}
	\item[矩阵] 
	\begin{equation}
		\begin{bmatrix}
			1 & x & 0 \\
			0 & 1 & -1
		\end{bmatrix}\begin{bmatrix}
			1  \\
			y  \\
			1
		\end{bmatrix}
		=\begin{bmatrix}
			1+xy  \\
			y-1
		\end{bmatrix}.
	\end{equation}
	
	\item[分段函数] 
	\[
	|x|=\begin{cases}
		x, & \text{if } x \geq 0,  \\
		-x, & \text{if } x < 0.
	\end{cases}
	\]
	
	\item[积分]
	\[
	\tilde f(\omega)=\frac{1}{2\pi}
	\int_{-\infty}^\infty f(x)e^{-i\omega x}, dx,
	\]
	
\end{description}
	
\section{图象}

\begin{figure}[h]
	\centering
	\includegraphics[scale=0.4]{example-report-gfx/eg_dist}
	\caption{报警次数的分布}
\end{figure}

\noindent
多图组合
\begin{figure}[h]
	\centering
	\begin{subfigure}{0.4\textwidth}
		\centering
		\includegraphics[width=\textwidth]{example-report-gfx/eg_dist}
		\caption{图1}
		\label{fig:image1}
	\end{subfigure}
	\quad 
	\begin{subfigure}{0.4\textwidth}
		\centering
		\includegraphics[width=\textwidth]{example-report-gfx/eg_dist}
		\caption{图2}
		\label{fig:image2}
	\end{subfigure}
	
	\vspace{6pt}
	
	\begin{subfigure}{0.4\textwidth}
		\centering
		\includegraphics[width=\textwidth]{example-report-gfx/eg_dist}
		\caption{图3}
		\label{fig:image3}
	\end{subfigure}
	\quad
	\begin{subfigure}{0.4\textwidth}
		\centering
		\includegraphics[width=\textwidth]{example-report-gfx/eg_dist}
		\caption{图4}
		\label{fig:image4}
	\end{subfigure}
	\caption{四张图}
\end{figure}

\noindent 旋转图片
\begin{center}
	\begin{tikzpicture}
		\node[rotate=30] at (0,0) {\includegraphics[width=120pt]{example-report-gfx/eg_dist}};
	\end{tikzpicture}
\end{center}
	
	
\section{Tikz 画图}

用 Tikz 画的流程图。

\input{example-report-gfx/eg_diagram}

\section{字体样式}

\begin{itemize}
	\item \textbf{人爽事成}
	\item \uline{划重点}
	\item \uuline{重点是划两次} 
	\item \uwave{波浪线是不是直线?}
	\item \sout{删掉但是看得见} 
	\item \xout{这就看得有点费劲了} 
\end{itemize}

\section{代码}

\noindent 行内代码 \lstinline{$ cat /etc/shells} \\

\noindent 命令行
\begin{lstlisting}[language=Bash]
$ cat /etc/shells
\end{lstlisting}

\begin{lstlisting}
$ git pull [<options>] [<repository> [<refspec>…​]]
\end{lstlisting}

\vspace*{2ex}

\noindent 代码块

\begin{lstlisting}[language=Python, caption={Python 示例}]
def factorial(n):  
    # Returns the factorial of n. 
    if n == 0:
        return 1
    else:
        return n * factorial(n - 1)

if __name__ == '__main__':
    result = factorial(5)
    print("The result is ", result)
\end{lstlisting}


\begin{lstlisting}[language=Java, caption={Java 示例}]
public class HelloWorld {
    public static void main(String[] args) {
        System.out.println("Hello World");
    }
}
\end{lstlisting}

\begin{lstlisting}[caption={伪代码}, 
	emph={[1]procedure, if, return, else, end}, 
	emphstyle={[1]\color{blue}}, 
	emph={[2]quickSort}, 
	emphstyle={[2]\color{red}}]
procedure quickSort(left, right)
    if right-left <= 0
        return
    else     
        pivot = A[right]
        partition = partitionFunc(left, right, pivot)
        quickSort(left,partition-1)
        quickSort(partition+1,right)    
    end if		
end procedure
\end{lstlisting}

\section{文本框}

朴素的文本框
\begin{tcolorbox}[
	colback=white, 
	colframe=gray, 
	sharp corners, 
	leftrule={2pt}, rightrule={0.5pt}, toprule={0.5pt}, bottomrule={0.5pt}, 
	left={2pt}, right={2pt}, top={3pt}, bottom={3pt}
	]
	tcolorbox 宏包是 Thomas F. Sturm 开发的一个用于绘制彩色文本框的宏包 tcolorbox 底层基于 pgf,功能也是十分强大。
\end{tcolorbox}

\noindent 带标题和颜色的文本框

\begin{tcolorbox}[
	colback=white, 
	colframe=blue!15, 
	sharp corners, 
	coltitle=black, 
	left={2pt}, right={2pt}, top={3pt}, bottom={3pt}, 
	title = {注意事项}]
	This is a \textbf{tcolorbox} with title.
	\tcblower
	Here, you see the lower part of the box.
\end{tcolorbox}
	
\end{document}